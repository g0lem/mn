
\documentclass[a4paper,12pt]{article}
\usepackage{graphicx,listings, amsmath}
\begin{document}
\begin{titlepage}
\title{Examen la Metode Numerice}
\author{Șerb Ștefan-Alexandru \\ 102F}
\date{21 Iunie 2022}
\maketitle
\end{titlepage}

\large \textbf{Problema 1}
\normalsize \bigskip


Pentru a evalua numeric $\int_{-\infty}^{+\infty}e^{-x^2}ln[2+ cos(x^2)]dx$
se considera formula de cuadratura Gauss-Hermite:


$$\int_{-\infty}^{+\infty}e^{-x^2}f(x)dx = \sum_{k=1}^{n}w_if_i 
+ \frac{n! \sqrt{\pi}}{2^n (2n)!}f^{(2n)}(c)$$

$$a<b<c$$

Unde $x_1 < x_2 < ..... < x_n$ sunt radacinile polinomului Hermite $H_n$

$$w_i = \frac{2^{n+1}n! \sqrt{\pi}}{[H_n'(x_i)^2]}$$
$$f_i = f(x_i), i = \bar{1,n} = (1,n) \cap Z$$

a) Pentru n = 4, gasiti o valoare aproximativa a integralei;

b) Care este valoarea ”exacta” a integralei (6 cifre semnificative)?

c) Dati o margine superioara a erorii pentru n=4. Comentati
rezultatele obtinute la punctele a-c;

d) Pentru a calcula o valoare aproximativa a integralei, cu o eroare
de $10^{-8}$, scrieti un program care implementeaza formula de
cuadratura Gauss-Hermite.

Observatie:

O sa notez in continuare valoarea termenilor $=^{not}T_{n}$ si valoarea sumelor
$=^{not}S_{n}$

\bigskip
\textbf{Subpunctul a:}
\normalsize \bigskip

Pentru n = 4 algoritmul va returna rezultatul ( folosind 8 zecimale dupa virgula ):

$$\int_{-\infty}^{+\infty}e^{-x^2}f(x)dx = \sum_{k=1}^{4}w_if_i 
+ \frac{n! \sqrt{\pi}}{2^n (2n)!}f^{(2n)}(c) + Rest$$


$$=w_if_i 
+ \frac{1! \sqrt{\pi}}{2^1 (2)!}f^{(2)}(c) + w_if_i 
+ \frac{2! \sqrt{\pi}}{2^2 (4)!}f^{(4)}(c) + w_if_i 
+ \frac{3! \sqrt{\pi}}{2^3 (6)!}f^{(6)}(c) + 
w_if_i 
+ \frac{4! \sqrt{\pi}}{2^4 (8)!}f^{(8)}(c) + R =$$

$$ = 1.76161047 + R = S_{4}$$

Pentru a vedea eroarea putem lua 


\newpage
\textbf{Subpunctul b:}
\normalsize \bigskip

Pentru o valoare mai exacta a acestei sume putem lua mai multi termeni,
de exemplu putem spune n=100

$$\int_{-\infty}^{+\infty}e^{-x^2}f(x)dx = \sum_{k=1}^{100}w_if_i 
+ \frac{n! \sqrt{\pi}}{2^n (2n)!}f^{(2n)}(c) + Rest \approx 1.78010061 = S_{100}$$

Aceasta este o valoare destul de "exacta"


Valoarea celui de-al 100-lea termen ($=^{not}T_{100}$) al acestei sume fiind de:

$$T_{100} = 4.02414459e-06 \approx 4 \cdot 10^{-6} $$

Putem afla acest termen din algoritmul nostru scazand din valoarea sumei
pentru n = 100 pe cea pentru n = 99

Pentru n = 100:

$$S_{100} = T_1 + T_2 + ..... + T_{99} + T_{100}$$
$$S_{99} = T_1 + T_2 + ..... + T_{98} + T_{99}$$

$$\Rightarrow T_{100} = S_{100} - S_{99}$$

\bigskip
\textbf{Subpunctul c:}
\normalsize \bigskip

Putem extinde putin ideea de mai sus, din moment ce aceasta serie trebuie
sa fie convergenta (altfel nu s-ar putea calcula).

Putem scrie ca:

$$S_{total} = S_{4} + Rest_1 = T_1 + T_2 + T_3 + T_4 + Rest_1$$
$$S_{total} = S_{5} + Rest_2 = T_1 + T_2 + T_3 + T_4 + T_5 + Rest_2$$

Egalam relatiile:

$$ T_1 + T_2 + T_3 + T_4 + Rest_1 = T_1 + T_2 + T_3 + T_4 + T_5 + Rest_2$$


Daca reducem toti termenii observam ca:

$$Rest_1 = T_5 + Rest_2$$

\newpage
Observam si ca acest lucru se poate intampla pentru orice n:

$$S_{total} = S_{n} + Rest_1$$
$$S_{total} = S_{n+1} + Rest_2$$
$$\Rightarrow Rest_1 = T_{n+1} + Rest_2$$


Prin convergenta acestui sir putem observa ca resturile se vor micsora,
astfel am putea aproxima eroarea scazand $S_{4}$ dintr-un S cu indice mult 
mai mare, de exemplu $S_{100}$

$$\varepsilon_4 \approx |S_{100} - S_{4}|$$

Si

$$S_{total} \approx S_{4} \pm \varepsilon_4$$

$$S_{100} = 1.78010061$$
$$S_4 = 1.7616104$$

Scazand:

$$\varepsilon_4 = 0.0184901380$$

Astfel, o sa iau marginea superioara a erorii:

$$\varepsilon = 0.02$$

$$S_{total} \approx S_4 \pm \varepsilon = 1.7616104 \pm 0.02$$

Aceasta margine de eroare este destul de buna, rezultatul acestei integrale 
conform Wolfram Alpha fiind: 

$$\int_{-\infty}^{+\infty}e^{-x^2}ln[2+ cos(x^2)]dx = 1.78009$$

Iar acesta apartine intervalului (1.7616104 - 0.02, 1.7616104 + 0.02)


\newpage
\textbf{Subpunctul d:}
\normalsize \bigskip

O scurta prezentare a algoritmului

\begin{lstlisting}
    function p=hermipol(n)

    k=2;
    
    p(1,1)=1;
    p(2,1:2)=[2 0];
    for k=2:n
       p(k+1,1:k+1)=2*[p(k,1:k) 0]-2*(k-1)*[0 0 p(k-1,1:k-1)];
    end
    
\end{lstlisting}

Aceasta functie genereaza polinomul Hermite


\begin{lstlisting}
    function I=gausshermi(f,a,b,n)

    p=hermipol(n);
    x=roots(p(n+1,:));
    
    G=feval(f,x);		
    
    
    for i=1:n
       C(i)=(2.^(n-1)*(factorial(n)).*sqrt(pi))./(n.^2.*(polyval(p(n,1:n),x(i))).^2);
    end
    
    
    I=dot(C,G);
    
    
    
\end{lstlisting}

Aceasta functie genereaza restul termenilor din ecuatie.

Apelez gausshermi folosind:

$$f(x) = ln[2 + cos(x^2)]$$
$$a = -\infty$$
$$b = +\infty$$

n reprezinta numarul de termeni pentru care calculam suma

\newpage
\large \textbf{Problema 2}
\normalsize \bigskip


Folosind metoda factorizarii LU a lui Doolittle, sa se rezolve sistemul liniar:

$$2x_1 - x_2 = 1$$
$$-x_1 + 2x_2 - x_3 = 0$$
$$-x_2 + 2x_3-x_4 = 0$$
$$-x_3 + 2x_4 = 1$$

Matricea sistemului este:
\[
    A = 
\begin{bmatrix}
    2 & -1 & 0 & 0 \\
    -1 & 2 & -1 & 0 \\
    0 & -1 & 2 & -1 \\
    0 & 0 & -1 & 2 \\
\end{bmatrix}
\]

Si matricea solutiilor:

\[
    b = 
\begin{bmatrix}
    1  \\
    0  \\
    0  \\
    1  \\
\end{bmatrix}
\]

Vom desparti matricea A in doua alte matrice folosind factorizarea LU. 
O matrice va contine toate elemente deasura diagonalei principale (inclusiv diagonala),
iar cea de a doua va contine toate elemente dedesubtul diagonalei principale inclusiv aceasta.

Dupa factorizare vom obtine rezultatul astfel:

\begin{lstlisting}
    D = L\b
    X = U\D
\end{lstlisting}

Aceasta este echivalentul urmatoarei operatii matematice:

$$D\cdot b = L \rightarrow D = L \cdot b^{-1} \Rightarrow$$

\begin{lstlisting}
    D = L\b
\end{lstlisting}

$$X\cdot D = U \Rightarrow X = U \cdot D^{-1} \Rightarrow$$

\begin{lstlisting}
    X = U\D
\end{lstlisting}


\end{document}